\begin{bibunit}

\addcontentsline{toc}{chapter}{Chapter 2: Materials and Methods}
\chapter*{Chapter 2\\ Materials and Methods}

\addcontentsline{toc}{section}{Abstract}
\section*{\begin{center}Abstract\end{center}}
Lorem ipsum dolor sit amet, consectetur adipisicing elit, sed do eiusmod tempor incididunt ut labore et dolore magna aliqua. Ut enim ad minim veniam, quis nostrud exercitation ullamco laboris nisi ut aliquip ex ea commodo consequat. Duis aute irure dolor in reprehenderit in voluptate velit esse cillum dolore eu fugiat nulla pariatur. Excepteur sint occaecat cupidatat non proident, sunt in culpa qui officia deserunt mollit anim id est laborum.

\vspace*{1.5cm}

\addcontentsline{toc}{section}{Introduction}
\section*{Introduction}
Lorem ipsum dolor sit amet, consectetur adipisicing elit, sed do eiusmod tempor incididunt ut labore et dolore magna aliqua. Ut enim ad minim veniam, quis nostrud exercitation ullamco laboris nisi ut aliquip ex ea commodo consequat. Duis aute irure dolor in reprehenderit in voluptate velit esse cillum dolore eu fugiat nulla pariatur. Excepteur sint occaecat cupidatat non proident, sunt in culpa qui officia deserunt mollit anim id est laborum \cite*{hashemloianAlienExoticAzolla2009, gufuGrowthReproductionFunctional2019}. Lorem ipsum dolor sit amet, consectetur adipisicing elit, sed do eiusmod tempor incididunt ut labore et dolore magna aliqua. Ut enim ad minim veniam, quis nostrud exercitation ullamco laboris nisi ut aliquip ex ea commodo consequat. Duis aute irure dolor in reprehenderit in voluptate velit esse cillum dolore eu fugiat nulla pariatur. Excepteur sint occaecat cupidatat non proident, sunt in culpa qui officia deserunt mollit anim id est laborum.

\section*{Initial Observations}
To determine the growth patterns of \textit{Azolla}, a small-scale observational experiment was conducted.

\vspace{1em}
\textit{Azolla} was harvested from the pond at the Plant Growth Facility at Macquarie University, Sydney, Australia (33.1923\textdegree S, 151.5749\textdegree E). Within that pond, three different, deliberately chosen locations were appointed to select a variety of \textit{Azolla} plants in different stages of maturity (Figure \ref{fig:MQ_pond}): a carpet at the edge of pond; a carpet in the middle of the pond; a low-density cluster in the middle of the pond. An additional 20L of pond water was extracted and passed through a coarse sieve to remove larger debris.

\begin{figure}[H]
  \centering
  \includegraphics[width = 35em]{img/pond.png}
  \caption{Pond at Macquarie University. A) A view of the pond. B) Carpet of \textit{Azolla} at the edge of the pond. C) Carpet of \textit{Azolla} in the middle of the pond. D) Low-density cluster of \textit{Azolla} in the middle of the pond.}
  \label{fig:MQ_pond}
\end{figure}

Non-BPA, UV resistant, food-grade 10L plastic tubs (internal dimensions approximately 346mm long x 242mm wide x 134mm deep, Appendix 1) were used to contain the treatments. Prior to use, the tubs were sprayed liberally with hydrogen peroxide (0.5\% w/w) then thoroughly rinsed with reverse osmosis (RO) water. Three tubs were each filled with 5L of the gathered pond water, 3 tubs were each filled with 5L of RO water. Five litres came to an average depth of 65mm in the centre of each tub.

\vspace{1em}
Seven fronds of \textit{Azolla} were selected by hand from the bucket in a hexagram pattern to alleviate selection bias (Figure \ref{fig:hexagram}). After these fronds were selected for a tub, the bucket’s contents were gently stirred using a flat wooden stick in a clockwise rotation for seven full revolutions to randomise the fronds’ locations for the next tub’s selection. Upon selection, each frond was inspected to ensure its maturity and root integrity (see Figure \ref{fig:frond_examples} for examples). Any fronds not found to be suitable were omitted. The 7 selected fronds were collectively placed onto the surface of the water at the centre of each tub.

\begin{figure}[H]
  \centering
  \includegraphics[width = 15em]{img/hexagram.png}
  \caption{Hexagram pattern. The 7 \textit{Azolla} fronds were selected from the edge of the bucket and at its centre by following the pattern laid out in this diagram.}
  \label{fig:hexagram}
\end{figure}

\begin{figure}[H]
  \centering
  \includegraphics[width = 30em]{img/frond_examples.png}
  \caption{Examples of selected fronds. A) Mature and healthy frond (included). B) Immature frond (omitted). C) Damaged frond (omitted).}
  \label{fig:frond_examples}
\end{figure}

The tubs were temporarily placed in a line in random order in the middle of a glasshouse. This glasshouse was shared with another researcher who required the following environmental conditions: CO2 at 400ppm, ambient lighting, day temperature at 24-26\textdegree C and night temperature at 18-20\textdegree C for 12 hours each.

\vspace{1em}
After 5 days, 20mL of a commercial-grade liquid fertiliser was added to each tub (For chemical composition, see Appendix 2). The tubs’ arrangement was randomised to account for any within-glasshouse effects.

\vspace{1em}
Whenever the tubs were moved, care was taken to minimise any turbulence of the water that might break the surface tension and damage the fronds.


\addcontentsline{toc}{section}{References}
\putbib

\end{bibunit}
